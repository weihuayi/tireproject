% !Mode:: "TeX:UTF-8"
\documentclass{book}
\input{en_preamble.tex}
\input{xecjk_preamble.tex}
\begin{document}
\title{轮胎数值模拟基础}
\maketitle

\chapter{前言}
\chapter{轮胎专业基础知识}

 外胎是由胎体、缓冲层(或称带束层)、胎面、胎侧和胎圈组成。外胎断面可分成几个单
独的区域:胎冠区、胎肩区(胎面斜坡)、屈挠区(胎侧区)、加强区和胎圈区。胎体:又称
胎身。通常指由一层或数层帘布层(具有强度、柔软性和弹性)与胎圈组成整体的(作为)充
气轮胎的受力结构。

轮胎通常由外胎、内胎、垫带3部分组成。也有不需要内胎的,其胎体内层有气密性好的橡
胶层,且需配专用的轮辋。世界各国轮胎的结构,都向无内胎、子午线结构、扁平(轮胎
断面高与宽的比值小)和轻量化的方向发展。

  外胎是由胎体、缓冲层(或称带束层)、胎面、胎侧和胎圈组成。 外胎断面可分成几
个单独的区域:胎冠区、胎肩区(胎面斜坡)、屈挠区(胎侧区)、加强区和胎圈区。

胎体又称胎身。通常指由一层或数层帘布层(具有强度、柔软性和弹性)与胎圈组成整体的(
作为)充气轮胎的受力结构。

  帘布层:是胎体中由并列挂胶帘子线组成的布层,是轮胎的受力骨架层,用以保证轮
胎具有必要的强度及尺寸稳定性。 *胎圈:轮胎安装在轮辋上的部分,由胎圈芯,帘布层
包边和胎圈包布等组成。它能承受因内压而产生的伸张力,同时还能克服轮胎在拐弯行驶
中所受的横向力作用,使外胎不致脱出轮辋。因此它必须有很高的强力,结构应紧密坚固
,不易发生变形。

  胎体需要有充分的强度和弹性,以便承受强烈的震动和冲击,承受轮胎在行驶中作用
于外胎上的径向、侧向、周向力所引起的多次变形。胎体由一层或多层挂胶帘布组成,这
些帘布能使胎体以及整个外胎具有必要的强度。

  缓冲层(或称带束层):斜交轮胎胎面与胎体之间的胶帘布层或胶层,不延伸到胎圈
的中间材料层。用于缓冲外部冲击力,保护胎体,增进胎面与帘布层之间的粘合。子午线
结构轮胎的缓冲层由于其作用不同,一般称为带束层。子午线轮胎胎面基部下,没胎冠中
心线圆周方向箍紧胎体的材料层。

  胎面:外胎最外面与路面接触的橡胶层(通常,把外胎胎冠、 胎肩:胎冠两侧的边缘
部分、胎侧、加强区部位最外层的橡胶统称为胎面胶)。

  胎面用来防止胎体受机械损伤和早期磨损,向路面传递汽车的牵引力和制动力,增加
外胎与路面(土壤)的抓着力,以及吸收轮胎在运行时的振荡。

轮胎在正常行驶时直接与路面接触的那一部分胎面称为行驶面。行驶面表面由不同形
状的花纹块、花纹沟构成,凸出部分为花纹块,花纹块的表面可增大外胎和路面(土壤)的
抓着力和保证车辆必要的抗侧滑力。花纹沟下层称为胎面基部,用来缓冲震荡和冲击

胎侧是轮胎侧部帘布层外层的胶层,用于保护胎体,又有弹性。

胎圈是轮胎安装在轮辋上的部分,由胎圈芯和胎圈包布组成,起固定轮胎作用。轮胎的规
格以外胎外径D、胎圈内径或轮辋直径d、断面宽B及扁平比(轮胎断面高H/轮胎断面宽 B)
等尺寸加以表示,单位一般为英寸(in)(1in=2.54cm)。

胎踵:胎圈外侧与轮辋胎圈座圆角着合的部分。

胎圈芯:由钢圈,三角胶条和胎圈芯包布制成的胎圈部分。*钢丝圈:有镀铜钢丝缠绕成的
刚性环,是将轮胎固定到轮辋上的主要部件。*装配线:模压在胎侧与胎圈交接处的单环或
多环胶棱,通常用以指示轮胎正确装配在轮辋上的标线。

\begin{itemize}
    \item 胎冠
    \item 胎肩
    \item 胎侧
    \item 趾口
    \item 钢丝圈
    \item 钢丝圈包布
    \item 胎体帘布(Carcass)
    \item 尼龙网布
    \item 缓冲层帘布(Breaker Carcass)
    \item 缓冲层垫胶(Breaker Rubber)
    \item 三角胶条(Bead Filler)
    \item 胎踵垫胶
    \item 胎肩胶
    \item 胎面胶
    \item 垫胶
    \item 胎侧胶(Sidewall Rubber)
    \item 子口胶
    \item 气密层
    \item 胎唇部(Bead)
    \item 花纹(Pattern)
\end{itemize}



\cite{iulian_rosu_finite_2018}
\bibliographystyle{abbrv}
\bibliography{ref}
\end{document}
